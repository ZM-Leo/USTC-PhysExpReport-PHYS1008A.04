\documentclass[a4paper]{extarticle}
\usepackage{ctex}
\usepackage{titlesec}
\usepackage{lipsum}
\usepackage{amsmath}
\usepackage{geometry}
\setCJKmainfont{STSong}[AutoFakeBold,AutoFakeSlant]
\setCJKsansfont{Microsoft YaHei}
\setCJKmonofont{KaiTi}
\setmainfont{Times New Roman}
\setsansfont{Times New Roman}
\setmonofont{Times New Roman}
\linespread{1.5}
\setlength{\parindent}{0pt}
\titleformat{\section}{\bfseries\fontsize{14pt}{\baselineskip}\selectfont}{\thesection}{0.5em}{}
\titleformat{\subsection}{\bfseries\fontsize{10.5pt}{\baselineskip}\selectfont}{\thesubsection}{0.5em}{}
\titleformat{\subsubsection}{\fontsize{10.5pt}{\baselineskip}\selectfont}{\thesubsubsection}{0.5em}{}
\geometry{left=2.5cm,right=2.5cm,top=2.5cm,bottom=2.5cm}
\everymath{\displaystyle}

\begin{document}
    \begin{center}
        \textbf{\fontsize{22pt}{\baselineskip} \selectfont 单摆法测重力加速度实验设计}\\
        \vspace{2em}
        \texttt{\fontsize{14pt}{\baselineskip} \selectfont 刘子墨 PB23000233}\\
        \texttt{\fontsize{14pt}{\baselineskip} \selectfont 少年班学院}\\
    \end{center}
    \section{实验原理}
    \hspace{2em}
    在实际的单摆实验中,悬线是一根有质量(弹性很小)的线,摆球是有质量有体积的刚性小球,摆角不为零,摆球的运动还受到空气的影响。
    此时单摆周期公式为:\\
    \begin{equation*}
        T=2\pi\sqrt{\frac{l}{g}\left[1+\frac{d^2}{20l^2}-\frac{m_0}{12m}\left(1+\frac{d}{2l}+\frac{m_0}{m}\right)+\frac{\rho_0}{2\rho}+\frac{\theta^2}{16}\right]}
    \end{equation*}
    若实验精度要求在$10^{-3}$以内,则这些修正项都可以忽略不计。在一级近似下,单摆周期公式为:\\
    \begin{equation*}
        T=2\pi\sqrt{\frac{l}{g}}
    \end{equation*}
    则重力加速度的测量公式为:\\
    \begin{equation*}
        g=\frac{4\pi^2l}{T^2}
    \end{equation*}
    \section{实验方案设计}
    \subsection{不确定度合成公式}
    \hspace{2em}
    根据测量公式,需测出摆长$l$和周期$T$。\\
    1、摆长$l$的不确定度
    \par\hspace{2em}
    摆长的测量模型为:\\
    \begin{equation*}
        L=\overline{L}+L_0
    \end{equation*}
    其中$\overline{L}$为摆长的平均值,$L_0$为钢卷尺的仪器误差对测量的影响。
    测量数据的平均值 $\overline{L}$ 的不确定度为标准差 $\mu_{\overline{L}}$ 。
    仪器误差B类正态分布,所以仪器误差 $L_0$ 的不确定度为 $\mu_{L_0}=\frac{\Delta_B}{3}=0.0667\,\text{cm}$ 。
    所以摆长的不确定度为:
    \begin{equation*}
        \mu_L=\sqrt{\mu_{\overline{L}}^2+\mu_{L_0}^2}=\sqrt{\mu_{\overline{L}}^2+\frac{\Delta_B^2}{9}}
    \end{equation*}
    2、周期$T$的不确定度
    \par\hspace{2em}
    周期的测量模型为:\\
    \begin{equation*}
        T=\overline{T}+T_0+T_\text{人}
    \end{equation*}
    其中$\overline{T}$为周期的平均值,$T_0$为秒表的仪器误差对测量的影响,$T_\text{人}$为人的反应时间。
    测量数据的平均值 $\overline{T}$ 的不确定度为标准差 $\mu_{\overline{T}}$ 。
    仪器误差B类正态分布,所以仪器误差 $T_0$ 的不确定度为 $\mu_{T_0}=\frac{\Delta_\text{秒}}{3}=0.0033\,\text{s}$ 。
    人的反应时间B类正态分布,所以人的反应时间 $T_\text{人}$ 的不确定度为 $\mu_{T_\text{人}}=\frac{\Delta_\text{人}}{3}=0.0667\,\text{s}$ 。
    所以周期T的不确定度为:
    \begin{equation*}
        \mu_T=\sqrt{\mu_{\overline{T}}^2+\mu_{T_0}^2+\mu_{T_\text{人}}^2}=\sqrt{\mu_{\overline{T}}^2+\frac{\Delta_\text{秒}^2+\Delta_\text{人}^2}{9}}
    \end{equation*}
    \par
    3、重力加速度的标准合成不确定度
    \par
    \hspace{2em}
    综上,重力加速度的标准合成不确定度为:
    \begin{equation*}
        \mu_g=\sqrt{\left(\frac{\partial g}{\partial l}\right)^2\mu_L^2+\left(\frac{\partial g}{\partial T}\right)^2\mu_T^2}=\sqrt{\left(\frac{4\pi^2}{T^2}\mu_L\right)^2+\left(\frac{-8\pi^2l}{T^3}\mu_T\right)^2}
    \end{equation*}
    \subsection{测量周期数量}
    \hspace{2em}
    已知单摆的摆长约为 70 cm,摆长测量的不确定度约为 0.2 cm,用秒表测量时间的不确定度约为 0.2 s。
    按照合肥市重力加速度为9.79 m/s$^2$,估计单摆周期约为$2\pi\sqrt{\frac{l}{g}}=1.68\,\text{s}$。
    如果测单个周期,由公式可知,重力加速度的标准合成不确定度为:
    \begin{equation*}
        \mu_g=\sqrt{\left(\frac{4\pi^2}{T^2}\mu_L\right)^2+\left(\frac{-8\pi^2l}{T^3}\mu_T\right)^2}
        =g\sqrt{\left(\frac{\mu_L}{L}\right)^2+\left(\frac{2\mu_T}{T}\right)^2}
        =2.33\,\text{m/s}^2
    \end{equation*}
    由此可知,测量一个周期时精度$\frac{\mu_g}{g}=0.238$,误差较大不满足实验要求。
    \par\hspace{2em}
    为了提高测量精度,可以使用累计放大法测量多个周期提高精度。设测量n个周期,测量总时间为T,则重力加速度测量公式变为:
    \begin{equation*}
        g=\frac{4\pi^2l}{\left(\frac{T}{n}\right)^2}=\frac{4\pi^2ln^2}{T^2}
    \end{equation*}
    重力加速度的标准合成不确定度为:
    \begin{equation*}
        \mu_g=g\sqrt{\left(\frac{\mu_L}{L}\right)^2+\left(\frac{2\mu_T}{T}\right)^2}
    \end{equation*}
    带入上述数据求得当$\frac{\mu_g}{g}\leq0.01$时,总时间T的最小值为41.74 s,即最少测量25个周期。
\end{document}